% --------------------------------------------------------------
% This is all preamble stuff that you don't have to worry about.
% Head down to where it says "Start here"
% --------------------------------------------------------------
 
\documentclass[12pt, twoside]{article}
 
\usepackage{fancyhdr}
\usepackage[margin=1in]{geometry} 
\usepackage{amsmath,amsthm,amssymb}

\usepackage{fancyvrb, newverbs}
\usepackage{xcolor}

\usepackage{enumitem}

\usepackage{graphicx}
% -------------------------------------------------------------
% Graphics path
% -------------------------------------------------------------
\graphicspath{{../graphs/}}

% -------------------------------------------------------------
% Verbatim background
% -------------------------------------------------------------
\definecolor{verbbg}{gray}{0.93}
\newverbcommand{\cverb}
  {\setbox\verbbox\hbox\bgroup}
  {\egroup\colorbox{verbbg}{\box\verbbox}}

% -------------------------------------------------------------
% Setup constants 
% -------------------------------------------------------------
\newcommand{\name}{Ethan Febinger, Ryan Mao \\ NetIDs: eef45, rym15}
\newcommand{\class}{CS 440: Intro to AI}
\newcommand{\hwTitle}{Project 2} % Change this for a new homework
\newcommand{\due}{March 22, 2021} % Change this for a new homework


% --------------------------------------------------------------
% Setup header and footer.
% --------------------------------------------------------------
\pagestyle{fancy}
\fancyhf{}
\fancyhead[C]{\hwTitle \hfill \thepage}
\fancyfoot[C]{\name \hfill \class}
\setlength{\headheight}{14.5pt} % set head height
\renewcommand{\headrulewidth}{0.4pt} % default is 0pt
\renewcommand{\footrulewidth}{0.4pt} % default is 0pt
 
\newcommand{\N}{\mathbb{N}}
\newcommand{\Z}{\mathbb{Z}}

\newcommand\ddfrac[2]{\frac{\displaystyle #1}{\displaystyle #2}}
 
\newenvironment{theorem}[2][Theorem]{\begin{trivlist}
\item[\hskip \labelsep {\bfseries #1}\hskip \labelsep {\bfseries #2.}]}{\end{trivlist}}
\newenvironment{lemma}[2][Lemma]{\begin{trivlist}
\item[\hskip \labelsep {\bfseries #1}\hskip \labelsep {\bfseries #2.}]}{\end{trivlist}}
\newenvironment{exercise}[2][Exercise]{\begin{trivlist}
\item[\hskip \labelsep {\bfseries #1}\hskip \labelsep {\bfseries #2.}]}{\end{trivlist}}
\newenvironment{problem}[2][Problem]{\begin{trivlist}
\item[\hskip \labelsep {\bfseries #1}\hskip \labelsep {\bfseries #2.}]}{\end{trivlist}}
\newenvironment{question}[2][Question]{\begin{trivlist}
\item[\hskip \labelsep {\bfseries #1}\hskip \labelsep {\bfseries #2.}]}{\end{trivlist}}
\newenvironment{corollary}[2][Corollary]{\begin{trivlist}
\item[\hskip \labelsep {\bfseries #1}\hskip \labelsep {\bfseries #2.}]}{\end{trivlist}}
 
\begin{document}
 
% --------------------------------------------------------------
%                         Start here
% --------------------------------------------------------------
 
\title{\hwTitle} % replace X with the appropriate number
\author{\name\\  % replace with your name
\class} % if necessary, replace with your course title
\date{\due}
 
\maketitle

\section*{Integrity}

We, Ryan Mao and Ethan Febinger, certify that the code and report for this assignment are our work alone.

\section*{Work Division}

    Ryan Mao: Problems 1-4, 6 

    \noindent Ethan Febinger: Problems 5, 7, 8


\tableofcontents
\vfill
\pagebreak
\section{Writeup}
\begin{enumerate}[itemsep=2mm,parsep=4mm]
    \item 
        \textit{Representation: How did you represent the board in your program, and how did you represent the information/ knowledge that clue cells reveal? How could you represent inferred relationships between cells?}
    
        The board was simply represented as a 2D matrix of cells. In this, each cell can be one of three things. If the cell is covered (i.e. the user/agent has not decided to query this cell), there will be a `?' at the corresponding location. If the cell has been queried before and is a mine, there will be a `M' present. If the cell has been queried and is safe (i.e. the cell itself is NOT a mine), there will be the appropriate clue value which describes the number of mines directly surrounding the current cell. 

        The knowledgebase consists of four 2D matrices. The first 2D matrix is the current board using the same representation methods discussed above. The second 2D matrix is the number of safe (non-mine) squares around a given cell. For example, if there are 2 mines around a cell with indices (i, j), then at indices (i, j) for this 2D matrix will be 6, as there are 6 cells around the current cell that are not mines. Our third 2D matrix in our knowledgebase is precisely the negation of the second 2D matrix. Examining the same cell as in the previous example, at indices (i, j) for the third 2D matrix would be a 2. Lastly, the fourth 2D matrix stores the amount of covered/unqueried cells around a current cell.

        The inferred relationships between cells is represented as sets. For example, if a cell has a clue value of 2, and none of the uncovered cells next to it are mines, then we know that the set of cells around the current cell has 2 mines in it. So, we say that $C = \{\text{number of nearby covered cells}\} = 2$. We then know any subset of this set has to have a value say $v \leq 2$. Thus, the $C - A = 2 - v$. From this, we can perform a sort of induction to aid our agent in the minesweeper program.

\end{enumerate}
 
\vfill
\pagebreak
\section{Non-Standard Libraries Used}

\begin{enumerate}
    \item
        \textit{matplotlib} - Generating and saving graphs.
    \item
        \textit{numpy} - arange function to generate equally separated float values between two floats.
        Was also used for generating curves of best fit, but is omitted from the submitted code.
\end{enumerate}
% --------------------------------------------------------------
%     You don't have to mess with anything below this line.
% --------------------------------------------------------------
 
\end{document}